% Titlepage
\title{ \huge{\textbf{Growth and Distribution of Vocabulary in Open Source Software}} \\[1.2cm]
\large{Faculty of Information and Communication Technologies\\
Swinburne University of Technology\\
Melbourne, Australia\\}
\vspace{1.2cm} 
\large{Submitted for the degree of Bachelor of Science (Professional Software Development) (Honours)} \\
\vspace{1cm} 
%\Large{Date} 
} 
\author{ \Large{\textbf{Allan Jones}} } 
\date{2010} 
\maketitle

%%% Abstract
\newpage 
\pdfbookmark[0]{Abstract}{abstract}
\chapter*{Abstract}
\vspace{-0.75cm}
\setstretch{1.12}

\textbf{TODO: Find a way to disambiguate here between natural language and programming language}
Software developers are afforded great flexibility in the terms of the language they can use to write source code, bound only by the syntactic rules which are governed by the platform they are developing for. This flexibility allows developers to expressive about what the software should do and the way in which it has been constructed. Typically, the language used is not arbitrarily selected, but rather, is a vocabulary of terms that maps to the developers understanding of the software system, which is generally shared by multiple parties.

As a software system is evolved over time, undergoing addition of new functionality and maintenance efforts, the vocabulary will also undergo change, being extended and modified to support these outcomes of evolution. This change threatens the understanding of the software held by developers, driving the need for control over the preservation of the vocabulary.

\textbf{TODO: Probably need a bit more here to communicate what is lacking in the current landscape, emphasising the need for research of this nature}

In this thesis, we address the problem of identifying when, in the evolution of a software system, efforts towards maintaining vocabulary are required and to which areas of the software they are best directed. We conduct a study of over 40 open source Java software systems, investigating how and when source code vocabularies are formed in the process of software evolution, the nature of their growth and change and how the terms in the vocabulary are distributed throughout the codebase. Finally, we determine the applicability of the laws of software evolution in describing the evolution of the source code vocabulary. 

\textbf{Note: From Raj -- revise}

Vocabulary that is formed early in the system's life is, in general, consistently the most popular and reveals the domain as well as the key design patterns in use. Although evolution motivates the creation of new terms, the vocabulary grows at a similar rate to as overall size. However, only rarely do new terms gain substantial popularity.

% Dedication
\newpage \vspace*{8cm} 
\begin{center}
	\large Dedications
\end{center}
\newpage

% Acknowledgements
\pdfbookmark[0]{Acknowledgements}{acknowledgements}
\chapter*{Acknowledgements}
\vspace{-0.5cm}

\vspace*{4cm}
Allan Jones, 2010

% Declaration
\chapter*{Declaration}
\vspace{-0.5cm}

\singlespacing

\tableofcontents \listoffigures \listoftables 
\newpage
